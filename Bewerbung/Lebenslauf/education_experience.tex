\begin{rubric}{\textcolor{black!20!blue!100}{praktische Tätigkeiten}}% Arbeitserfahrung
%\subrubric{Research}%Forschung
	\subrubric{Student der Universität Stuttgart}
		\entry*[\hphantom{00/0000 --- 00/}2016]
			Bachelorarbeit: Optimierte Schwarz Methode für Magnetostatik
			
			\setlength{\hangindent}{\widthof{$\rightarrow$ }}
			$\rightarrow$ Simulationssoftware: COMSOL
			
			$\rightarrow$ Simulationsmethode: FEM
		\entry*[\hfill 2015]
			Praktische Übungen im Labor: zirkulare Gaußkanone
			
			\setlength{\hangindent}{\widthof{$\rightarrow$ }}
			$\rightarrow$ Simulationssoftware: COMSOL
			
			$\rightarrow$ Simulationsmethode: FEM
			
			$\rightarrow$ Optimierung der Spuleneigenschaften: Gradientenverfahren
		\entry*[\hfill 2015]
			Physikalisches Praktikum I: praktische Grundlagen wissenschaftlicher Experimente
			
			\setlength{\hangindent}{\widthof{$\rightarrow$ }}
			$\rightarrow$ Erfassung von Rohdaten durch Messungen
			
			$\rightarrow$ Auswertung und Fehlerrechnung
			
			$\rightarrow$ wissenschaftliche Protokollierung
		\entry*[\hfill 2014]
			Teamarbeit: lineare Gaußkanone
			
			\setlength{\hangindent}{\widthof{$\rightarrow$ }}
			$\rightarrow$ Programmierung Mikrocontroller mit C
			
			$\rightarrow$ Geometriemodellierung und Auswertung einer Simulation mit HyperMesh
			
			$\rightarrow$ Optimierung der Spuleneigenschaften: Gradientenverfahren
			
			$\rightarrow$ Spulenwicklung
\end{rubric}