		seit August 2025 bin ich arbeitslos und suche nach Arbeit.
		Weiterhin bin ich überwiegend ohne festen Wohnsitz und obdachlos seit Frühling 2019.
		Ich schlafe gerne im Zelt, im Auto, in Anhänger, bei Freunden und Familie, im Erfrierungsschutz für Obdachlose und in Obdachlosenunterkünften.
		Ich habe kein Wunschgehalt, um mir mit Geld nach Feierabend Emotionen zu kaufen, stattdessen strebe ich an, während der Arbeit mit Emotionen bezahlt zu werden.
		Ich fordere, dass Arbeiten unter Einfluss von legalen Drogen wie Alkohol und Exorphinen am Arbeitsplatz nicht erlaubt ist.

		Wenn der Arbeitsweg oder die Wohnung zu viel Zeit oder Geld im Vergleich zur Arbeit in Anspruch nimmt, müssen Sie davon ausgehen, dass ich in der Nähe der Arbeit im Auto schlafen werde.
		In diesem Fall können Sie davon ausgehen, dass ich die Küche in der Firma gut benutzen werde.
		Gibt es keine Küche in der Firma, werde ich auf dem Parkplatz kochen.
		Wird meine Arbeit einigermaßen gut bezahlt, kann ich mir einen Wohnwagen oder ein Wohnmobil leisten.
		Wird meine Arbeit sehr gut bezahlt, kann ich mir eine kleine Wohnung in der Nähe der Firma leisten.
		Im Optimalfall erlauben Sie mir mit meinem Schlafsack in der Firma zu schlafen.
		Schläft man als Team gemeinsam in der Firma, stärkt das den Teamgeist, die Arbeitsmotivation und die Zusammenarbeit.
		Ist der Ruf erst einmal ruiniert, arbeitet es sich ganz entspannt.

		Der Hintergrund ist, dass die Wohnungen heutzutage überteuert sind für die Leistung, die man bekommt.
		Es stresst mich, wenn ich für Vermieter arbeite, die sich nicht wirklich um die Anliegen der Mieter kümmern.
		Das führt zu weniger Lebensmotivation und dadurch zu weniger Produktivität bei der Arbeit.
		Momentan beispielsweise wohne ich in einer Wohnung, in der die Zentrallüftung für das Bad sehr laut ist und ein permanentes starkes Hintergrundrauschen erzeugt und ich kann es nicht abstellen, denn es ist eine aktive Zentrallüftung (mit Ventillator o.ä.).
		Dieses permanente Hintergrundrauschen stellt eine Belastung für die Psyche dar, die aus meiner Sicht nicht sein muss.
		Das Gebäude scheint eine Fehlkonstruktion zu sein und anstatt dass man dafür bezahlt wird diesen psychischen Druck auszuhalten, muss man noch extra dafür bezahlen.
		Ich habe einen Antrag auf Bürgergeld gestellt und die Wohnung wird voraussichtlich von Bürgergeld bezahlt, deshalb kann ich es mir psychisch leisten, in einer schlechten Wohnung zu wohnen.
		Das ändert sich, sobald ich selber mit meiner Arbeitskraft die Wohnung bezahlen muss.

		Im Mai 2024 sollte ich bei Lieferando in Frankfurt anfangen.
		Ich bekam einen Arbeitsvertrag, jedoch bekam ich keine Arbeit.
		Ich habe auf Einweisungstermine gewartet und regelmäßig nachgefragt, woran es liegt, dass ich keine Termine bekomme.
		Am 12.06.2024 kam die abschließende Antwort, dass der Vertrag rückwirkend aufgehoben wird, mit der Begründung, dass der Betriebsrat gegen meine Einstellung gestimmt hat.
		Es war in Artikel 11 des Arbeitsvertrags festgehalten, dass das Arbeitsverhältnis unter der aufschiebenden Bedingung steht, dass der Betriebsrat meiner Einstellung zustimmen muss.
		Das hat zu diversen Komplikationen geführt.

		Im Oktober 2025 sollte ich bei ANT Transport GmbH in Moosburg anfangen.
		Ich bekam einen Arbeitsvertrag, jedoch bekam ich keine Arbeit.
		Stattdessen erhielt ich am 26. September 2025 die Mitteilung, dass der Vertrag gekündigt wird.
		Das hat zu diversen Komplikationen geführt.

		Um dies in Zukunft zu vermeiden, fordere ich eine Kaution in Höhe von 1000 € für entstandene Ausgaben aufgrund solcher Komplikationen, die nach erfolgreichem Abschluss einer Probezeit von mindestens drei Monaten zurückerstattet wird.
		Ich habe kein Rücklagenkapital, um für aufgrund solcher Komplikationen entstandene Ausgaben aufzukommen.

		Bitte beachten Sie, dass in meinem Lebenslauf nur Projekte/Arbeitsverhältnisse aufgelistet sind, mit denen ich auch Geld verdient habe.
		Ich habe im privaten Bereich schon viele kleine Projekte realisiert mit verschiedenen Sprachen und Tools (diese Bewerbung ist nahezu vollständig durch Shell-Skripte, LaTeX und Python auto-generiert).

		Ich bin ein flexibler Mensch und mag generell Probleme techninsch zu lösen, Systeme zu optimieren und neue Dinge zu entdecken, verstehen und zu lernen.
		Neuen Herausforderungen stehe ich offen gegenüber.
		Aufgrund meiner schnellen Auffassungsgabe arbeite ich mich problemlos in mir unbekannte Aufgabenbereiche ein.
		Auch bei hohem Arbeitsaufkommen behalte ich stets den Überblick und arbeite gleichbleibend konzentriert und zielorientiert.
		Jedoch verliere ich bei hoher psychischer Belastung meine Leistungsfähigkeit.
		Ich arbeite ohne Stress.
		Unter Stress und Zeitdruck macht mein Darm Probleme und ich leide psychisch und gesundheitlich stark.
		Insgesamt bin ich für Tätigkeiten geeignet, die viel Einarbeitung erfordern und hohe Komplexität besitzen oder für knifflige Probleme, an die sich nur die wenigsten ran trauen.

		In meiner langjährigen Erfahrung als Lieferfahrer bei verschiedenen Unternehmen wie Amazon Flex (2020–2022), Lieferando (2020–2021), Durstexpress (2020), Wolt Logistics (2024) und flaschenpost (2024) habe ich umfassende Kenntnisse in der effizienten Auslieferung von Paketen, Speisen und Getränken gesammelt.
		Dabei lag mein Fokus stets auf Pünktlichkeit, Kundenzufriedenheit und der sicheren Handhabung von Waren – sei es mit dem eigenen Pkw oder Roller.
		Ich bin mit der Nutzung von Navigations-Apps, der Optimierung von Routen und dem Umgang mit zeitkritischen Lieferungen bestens vertraut, was mir ermöglicht, auch unter hohem Arbeitsdruck ruhig und zielorientiert zu arbeiten.
		Diese Praxiserfahrung macht mich zu einem zuverlässigen Teammitglied für Ihren Lieferdienst, wo ich gerne meine Flexibilität und mein Engagement einbringe.

		Durch meine mehrjährige Erfahrung als Softwareentwickler habe ich viel über den praktischen Workflow in der Softwareentwicklung lernen dürfen.
		Meine Erfahrung in der Softwareentwicklung setzt sich hauptsächlich aus allgmeiner Entwicklung zusammen, bei der ein bestehendes System differenziell verändert wurde, um es zu verbessern, z. B. durch das Lösen von Bugs oder das Hinzufügen und Ändern von Features.

		Ich kenne mich mit den gängigsten objektorientierten Sprachkonzepten, Softwareentwicklungstools (Versionierungs-, IDE-, CI-Tools) und Vorgehensweisen (TDD, Scrum) aus.

		Auf eine Einladung zu einer Verhandlung über einen Praktikumsvertrag, Ausbildungsvertrag, Arbeitsvertrag oder Projektvertrag (als Freiberufler) würde ich mich freuen.